Entusiasmo, empolgação, regalias, são alguns dos fatores que levam uma pessoa a participar de hackathons, eventos baseados em duração temporal onde os participantes buscam superar-se e superar seus adversários e colegas através de desafios que geralmente duram um final de semana carregados de brindes e diversão. Mas será que todos os participantes estão atentos aos regulamentos e até onde eles têm prioridade em seus direitos? Através deste estudo buscamos entender a percepção dos competidores das hackathons a fim de compreender melhor o entendimento e compreensão sobre os seus direitos à propriedade intelectual em determinada hackathon. Através de entrevistas semi-estruturadas e uma \textit{survey} para melhor compreender as percepções e entendimentos dos participantes. Como resultado ainda existe um relaxamento quanto a clareza e melhores definições nos regulamentos, assim levando o participante a não compreender a cessão ou não cessão dos direitos à PI em uma determinada hackathon.  


\begin{keywords}
Hackathon, Propriedade Intelectual, Crowdsourcing, Time-Bounded Events, Percepção
\end{keywords}