\chapter{Conclusão}
\label{chp:conclusão}


Os eventos de hackathons são cercados de diversas regalias onde os participantes podem aproveitar, e muito, a sua estadia com um certo conforto, dependendo dos organizadores. Assim, muitas vezes, assuntos pertinentes ao bem estar ou relacionados aos direitos que o participante tem passam despercebidos. \citet{zukin2017hackathons} cita diversos pontos em que os participantes, em meio a empolgação do evento, negligenciam, ou lembram de forma mínima. Um deles é o tópico sobre com quem fica a PI do código produzido no evento, onde em alguns casos é dúbia a definição. \citep{steele_2013}

Como pôde ser visto nos resultados, a clareza e melhores definições nos regulamentos podem sim ajudar o participante a entender melhor o que significa a cessão ou não cessão dos direitos à PI em uma determinada hackathon. Com isso, dá-se mais liberdade ao participante de escolher ou não ceder os seus direitos. A lei brasileira é clara ao dizer a quem pertence os direitos à PI em um determinado evento, porém os regulamentos, muitas vezes mal escritos, põem em dúvida a cessão ou não dos direitos pelo participante, como já ocorreram em regulamentos bizarramente escritos\footnote{Regulamento retirado do ar, mas pode ser encontrado em: <https://docplayer.com.br/23433151-Regulamento-hackathon-gerdau-2016.html>. Acessado em 20 de mai. 2020} \footnote{Notícia pode ser encontrada em: <https://meiobit.com/86685/concurso-desenvolvedores-app-samsung-smarttv/>. Acessado em 20 de mai. 2020}.

Conforme pesquisa os participantes desejam sim, aprender mais sobre seus direitos à PI e possuem interesse no mesmo. A partir do momento em que um participante começa a compreender melhor o conceito de propriedade intelectual, os direitos que o mesmo tem ao escrever um pedaço de código, ao expor uma ideia de forma visível e fora da sua mente, o mesmo consegue escolher melhor o evento que participará, não será ludibriado por atrativos. Por isso a importância de aprender e entender os seus direitos e os direitos a um código-fonte escrito por um participante ou sua equipe.

O entendimento sobre o assunto fará com que as hackathons sejam mais acessíveis, além de garantir aos participantes liberdades de escolha, principalmente escolher ou não ceder os direitos à PI em um determinado evento. Não apenas participar do evento por uma empolgação e esquecer aquilo que pode beneficiar o participante, juntamente com a empresa patrocinadora do evento, de forma cooperativa e saudável.




\section{Trabalhos Futuros}


 Este é um trabalho que pode ser estendido amplamente, onde pode ser abordado o mesmo tema em caráter mais abrangente nacionalmente e também internacionalmente visto a quantidade restrita de estudos do assunto. Além de também abordar outros aspectos de hackathons envolvendo a provável insalubridade dos eventos.