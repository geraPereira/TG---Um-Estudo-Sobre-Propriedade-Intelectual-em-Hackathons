\chapter{Discussão}
\label{chp:Discussão}

A empolgação e entusiasmo levam jovens e adultos a participarem de eventos de hackathons \citep{zukin2017hackathons}, os mesmos tendem muitas vezes a negligenciar certos aspectos, tais como bem estar ou mesmo direitos à propriedade intelectual como \citet{zukin2017hackathons} descrevem, pois muitas vezes existe uma certa efevercência e empolgação, além de uma grande interação que os levam a negligenciarem esses pontos inerentes a eles \citep{zukin2017hackathons} \citep{pompa2013allen}. Por sua vez, muitas empresas aproveitam essa situação de fragilidade para monetizar os códigos dos participantes. \citep{pompa2013allen}

Para entender melhor a situação de como os participantes de hackathons veem e compreendem sobre os seus direitos ao ingressarem em uma competição, foi realizado este estudo. 


%% PERCEPÇÂO ESTUDO QUALITATIVO
Alguns pontos do estudo qualitativo, entender a percepção em alguns tópicos abordados como a percepção do entendimento sobre PI pelos regulamentos das competições que eles participaram. Onde os participantes entendem que PI é algo inerente a eles, onde eles sabem que aquela determinada ideia pertence a eles. Quanto ao nível de prioridade e a importância que a organização do evento dá ao tema os entrevistados não veem com clareza que o tema dos direitos à PI é bem abordado nos regulamentos. Eles têm a impressão que as empresas focam muito no produto final e não nesses tópicos de tamanha relevância para os participantes.

%%%%%%%%% RANKING
Foi pedido para os entrevistados enumerarem de 1 a 5 onde 1 é o mais importante e 5 o menos importante, quais critérios eles consideram como mais importante dentre os seguintes: Critério de avaliação, premiação,direitos à propriedade intelectual, organização da equipe, programação. Foi visto que todos tiveram uma certa divergência quanto aos principais pontos, onde foi visto que os entrevistados 1 e 3 escolheram o tópico de premiação como mais importante. Diferente do visto na pesquisa quantitativa que coloca como mais importante o tópico de critério de avaliação e uma pouca relevância ao tópico de premiação.


Também perguntados sobre a clareza quanto a quem fica os direitos à PI do conteúdo feito por eles naquele evento, os entrevistados não encontraram clareza quanto ao conteúdo sobre PI. O entrevistado 2 fala: \begin{quote}
    "Muitas das hackathons não deixa claro o suficiente... deixa claro a premiação, tal.[...]". 
\end{quote}
Somente o entrevistado 4 viu uma certa clareza, porém de forma negativa \begin{quote}
    "Fica claro que não é meu, geralmente todo conteúdo produzido numahackathon, é da hackathon[...]". 
\end{quote}

Quanto ao sentimento dos participantes em relação ao respeito do direto à PI há divergências entre os participantes de ambas as categorias, levando a abrir cada vez mais o debate. Pelo lado dos que \textbf{cederam PI}, o entrevistado 1 sente que o seu direito está sendo colocado de forma limitada, pois o pensamento é um pouco mais voltado ao lucro das ideias, enquanto o entrevistado 2 tem a percepção de mais liberdade quanto ao respeito dos seus direitos. Já pelo lado dos que \textbf{não cederam PI} a entrevistada 3 "não parou pra pensar sobre o assunto", enquanto o entrevistado 4 nota que o direito à PI vai ser respeitado de acordo com a escolha do participante em entrar ou não na competição. \begin{quote}
    "Como a pessoa escolhe participar se quiser, sim é respeitado.  O direito você que vai dizer se vai querer ou não".
\end{quote}

Quando perguntados se eles venderiam ou não eles foram categóricos afirmando que venderiam, ambos os entrevistados que \textbf{cederam PI} venderiam por um valor acima do oferecido no regulamento da hackathon. Os entrevistados que \textbf{não cederam PI} venderiam com ressalvas, principalmente no sentido de analisar melhor as propostas do mercado.

Ao serem perguntados sobre as principais diferenças encontradas nos regulamentos dentre as hackathons que participaram, ambos os participantes que \textbf{cederam PI} falam da falta de clareza entre os regulamentos, enquanto o entrevistado 4 focou no objetivo de privatização das ideias pelas hackathons do tipo empresariais.

Ao medir a confiança dos participantes com relação às hackathons, perguntamos se os entrevistados deixariam de participar de uma hackathon caso o direito à propriedade intelectual não ficasse com eles. A premiação seria o ponto chave para que eles decidissem não participar, como a entrevistada 3 pontua \begin{quote}
    "Por que a premiação é meio que um pagamento pela sua ideia,então você tá abrindo mão da sua ideia por um valor".
\end{quote}

Quando perguntados se teriam confiança em ceder os códigos-fonte que criaram para a empresa organizadora os participantes que cederam PI foram mais conservadores, cederiam com menos impedimentos dependendo da relevância ou dependendo do impacto que a solução poderia causar, diferente dos participantes que não cederam PI, que se mostram mais abertos cederiam com mais facilidade.

Ao serem indagados se nos eventos de hackathon, sentem que o seu direito à propriedade intelectual é respeitado, os entrevistados que \textbf{cederam PI} não sentem ou possuem dúvidas quanto ao respeito dado pelas organizadoras do evento ao seu direito à PI. Contrariando os participantes que \textbf{não cederam PI} que afirmam que houve respeito sim, aos seus direitos à PI.

















%% PERCEPÇÂO ESTUDO QUANTITATIVO

Também foi realizada uma pesquisa quantitativa visando entender melhor os pontos de vista de forma mais ampla. Assim, no presente estudo pudemos destrinchar as respostas das perguntas realizadas.

No que se diz respeito a leitura dos regulamentos 66\% responderam que leem, mostrando que há um interesse em entender em que tipo de evento estão participando. 

Questionando os participantes para entender qual tópico eles acham mais interessante, pudemos perceber que quase em sua maioria os participantes buscam o tópico de \textbf{premiação}, com 27,94\%, por achar interessante saber o quanto podem ser estimulados participando do evento. \textbf{Critério de avaliação} ficou em segundo lugar com cerca de 16,18\%, o que nos leva a refletir que o participante quer entender como chegar nessa premiação.

%ranking polêmico
Na pergunta onde os entrevistados enumeraram de 1 a 5 onde 1 é o mais importante e 5 o menos importante, quais critérios eles consideram como mais importante dentre os seguintes: Critério de avaliação, premiação,direitos à propriedade intelectual, organização da equipe, programação, os participantes elegeram o tópico de critério de avaliação como o tópico mais importante ao ler o regulamento, onde o mesmo é relevante para 55,12\%, enquanto o tópico de direitos à PI é relevante para 50\% dos participantes, formação de equipe é relevante para 37,18\%, programação do evento 21,27\% e a premiação é relevante para 36,30\% dos entrevistados.
%%%%%%%%%%%%%%%%%%%%%%%%%%%%%%

Quando perguntado o quão importante é o tópico sobre PI nos regulamentos das hackathons, essa que foi uma pergunta com resposta em formato da escala de Likert, em sua maioria, cerca de 69,23\% dos participantes afirmaram ser importante enquanto outros 30,76\% se mostraram neutros ou não consideram importante o tópico.

Com relação a clareza, 35,06\% afirmaram que o regulamento deixa claro pra quem fica os direitos à PI enquanto 64,94\% declararam neutralidade ou a não clareza dos regulamentos.

Cerca de 53,6\% afirmaram que cedem os direitos à PI numa hackathon. O que nos leva a pergunta quanto a clareza dada nos regulamentos conforme visto no tópico anterior. Como não há clareza nos regulamentos, como entender a cessão ou não cessão dos direitos à PI em um determinado evento? Com quem realmente fica? 

Assim como foi visto, 90\% dos participantes afirmam que venderiam o protótipo da solução. E como pode ser visto na pergunta seguinte, o valor a ser pago, de acordo com 57\% dos entrevistados seria de 1 a 10 vezes o valor da premiação, 6\% doariam a solução enquanto outros 37\% venderiam o protótipo por um valor no mínimo 11 vezes o valor da premiação, assim corroborando com os entrevistados da pesquisa qualitativa em que apenas um dos entrevistados venderia a solução por preços extremamente acima dos valores propostos pelas organizações das hackathons.

Em sua maioria, os participantes, com cerca de 55,9\%, deixaria sim, de participar de uma hackathon caso o direito à PI não ficasse com ele naquele evento. Assim como 49,35\% afirmaram que já abriu mão de PI em algum hackathon, enquanto 50,65\% afirmam não terem aberto mão. O resultado com pouca disparidade leva a crer na desconfiança dos participantes quanto a clareza nos regulamentos vistos em tópicos anteriores.

Quando indagados sobre o que o participante sentem quanto ao respeito os direitos à PI nos eventos 46,76\% sentem que são respeitados enquanto outros 53,25\% se mostram neutros ou não sentem que seus direitos são respeitados.

Também foi visto o quão interessante o participante acha saber ou entender sobre o assunto de PI, onde massivamente 79,22\% veem com bons olhos e interesse no assunto, mostrando o desejo em aprender cada vez mais sobre algo de tamanha relevância. 

Por fim os participantes indagam que há dúvidas quanto a ceder ou não os códigos-fonte à empresa organizadora do evento após a hackathon, pois vemos que 49,35\% dos participantes não confiam em ceder enquanto outros 50,65\% cederiam sem problemas algum.


\section{Limitações do Trabalho}

Foi constatado ao final da pesquisa, na realizada com abordagem quantitativa que, em sua maioria, 59\% dos participantes, residem no estado de Pernambuco, mais precisamente todos residentes de municípios que fazem parte da Região Metropolitana do Recife. Assim como na pesquisa qualitativa, todos os participantes também residem em municípios que fazem parte da Região Metropolitana do Recife. Assim o escopo da pesquisa pode possuir um viés local.
