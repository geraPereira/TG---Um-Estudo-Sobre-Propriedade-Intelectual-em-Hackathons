\chapter{Introdução}
\label{chp:introdução}

Eventos de programação voltados a criar soluções inovadoras para um determinado problema específico estão na moda. A busca por soluções otimizadas e cada vez mais econômicas motiva os  organizadores a idealizarem desafios cada vez mais inovadores nas hackathons.


Em termos leigos pode-se dizer que hackathons são eventos de maratonas hackers de exploração de novas ideias e produtos, em sua maioria voltados para tecnologia e/ou questões sociais. \citep{briscoe2014digital}

Desenvolvedores de software, designers e outros profissionais relacionados à área de programação participam desses eventos que trazem diversos benefícios às empresas, quer sejam \textit{startups} ou empresas tradicionais em busca de inovação. Auxiliando-as no desenvolvimento de novas tecnologias e apresentando novos talentos.

\citet{zukin2017hackathons} cita diversos pontos em que os participantes, em meio a empolgação do evento, negligenciam, ou lembram de forma mínima. Um deles é o tópico sobre com quem fica a PI do código produzido no evento, onde em alguns casos é dúbia a definição. \citep{steele_2013}

\citet{zukin2017hackathons} também afirma que em meio a tanta inovação, recreação e oportunidades de crescer o portfólio individual de cada participante, existe um lado negro muito forte, onde os participantes lutam por premiações abaixo do valor necessário, passam dias confinados em ambientes insalubres. É visto como estratégia para empresários da tecnologia explorarem os participantes em busca de soluções baratas e rápidas.


Para melhorar o entendimento dos participantes foi criado o \textit{The Hack Day Manifesto}\footnote{The Hack Day Manifesto <https://hackdaymanifesto.com/>}, onde informa sobre alguns requerimentos básicos que ajudam no sucesso da Hackathon. Um guia completo e sugestivo, com detalhes, para evitar dualidades de entendimento no evento e em seu regulamento.

No Brasil, no ano de 2016, ocorreu uma hackathon promovido pela empresa Gerdau. O qual chamou muito a atenção para alguns pontos que foram tratados como "pegadinha" dos organizadores para os participantes mais desavisados. 

Na 10ª cláusula, que trata da propriedade intelectual lê-se no segundo item:
\begin{quote}
    "Os participantes se obrigam a ceder à Gerdau, de forma não onerosa e exclusiva, todos os direitos patrimoniais decorrentes dos projetos desenvolvidos durante o Evento, sempre reservando para si os direitos morais a eles associados. Os participantes ainda declaram ser livres de quaisquer ônus ou embaraços à Gerdau os direitos patrimoniais ora cedidos."\footnote{Regulamento retirado do ar, mas pode ser encontrado em: <https://docplayer.com.br/23433151-Regulamento-hackathon-gerdau-2016.html>. Acessado em 20 de mar. 2020}
\end{quote} 
Onde todos os direitos sobre as ideias desenvolvidas são de propriedade da Gerdau e não dos desenvolvedores.

Por fim, o quarto item diz:
\begin{quote}
    "Os participantes serão responsáveis pela utilização de suas ideias e pela elaboração de seu projeto, assumindo total e exclusiva responsabilidade decorrente de eventuais reivindicações de terceiros relativos a direitos de propriedade intelectual ou de direito autoral, sendo preservado o direito de regresso ou de denunciação à lide em razão de eventuais questionamentos de terceiros."\footnote{Regulamento retirado do ar, mas pode ser encontrado em: <https://docplayer.com.br/23433151-Regulamento-hackathon-gerdau-2016.html>. Acessado em 20 de mar. 2020}
\end{quote}

Os desenvolvedores responsabilizam-se legalmente sobre suas criações. A Gerdau exime-se de quaisquer responsabilidade e problemas posteriores, tais como acusações de plágio, onde os desenvolvedores responderão pelos atos.

A Gerdau, neste caso, utilizou o nome Hackathon para absorver ideias dos participantes sem custo, não iria pagar direitos de propriedade intelectual e nem de imagem, e caso haja algum problema legal, quem responderia seriam os criadores e não a empresa, sem deixar muito claro aos participantes, principalmente os mais desavisados.

Este trabalho tem como objetivo analisar e compreender como a propriedade intelectual é percebida pelos participantes de hackathons e como as empresas expressam em seus regulamentos a quem pertence os direitos de propriedade intelectual. Para isso foram realizadas surveys e entrevistas semi-estruturadas como forma de coleta de dados a respeito dos participantes e análise de regulamentos de hackathons. Diante do objetivo geral serão realizadas pesquisas quantitativas e qualitativas sobre o assunto e serão propostas formas de como melhorar a qualidade e percepção da propriedade intelectual para os competidores.