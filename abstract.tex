Enthusiasm, excitement, perks, are some of the factors that lead a person to participate in hackathons, time-bounded events where participants seek to overcome themselves and overcome their opponents and colleagues through challenges that usually last a weekend laden with gifts and fun. But are all participants aware of the regulations and how far do they have priority over their rights? Through this study we seek to understand the perception of hackathons' competitors in order to better understand the knowledge and comprehension of their intellectual property rights in a given hackathon. Through semi-structured interviews and a survey to better understand the participants' perceptions and comprehension. As a result, there is still a relaxation in terms of clarity and better definitions in the regulations, thus leading the participant to not understand the assignment or non-assignment of rights to IP in a given hackathon.

\begin{keywords}
Intelectual Property, Hackathon, Time-bounded Events, Crowdsourcing, Perception
\end{keywords}